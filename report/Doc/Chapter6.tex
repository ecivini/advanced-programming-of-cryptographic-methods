\chapter{Instructions for installation and execution}

This chapter provides instructions for installing, configuring, and executing the 
Certificate Authority. These instructions are designed to enable readers to successfully 
deploy and test the system in their own environments.

\section{Software prerequisites}

The following software must be installed on the target system:
\begin{itemize}
    \item \textbf{Operating System}: Linux, macOS, or Windows with WSL2 (tested on Linux)
    \item \textbf{Docker}: Version 28.0 or higher (tested with 28.3)
    \item \textbf{Docker Compose}: Version 2.37 or higher (tested with 2.37.3)
    \item \textbf{Git}: Version 2
    \item \textbf{Firefox}: Version 133 or higher (for the web interface, other browsers may not work correctly)
\end{itemize}

\section{Installation process}

\subsection{Environment setup}
There are two ways to set up the project: via git or by downloading the zip archive. 
\begin{verbatim}
# 1) Verify Docker and Docker Compose are installed
$ docker --version
Docker version 28.3.0, build 38b7060a21

$ docker compose version
Docker Compose version 2.37.3

# 2.a) (Git approach) Clone the repository 
$ git clone https://github.com/ecivini/advanced-programming-of-cryptographic-methods.git

# 2.b) (Zip approach) Download and unzip the project archive
$ unzip apocm.zip

# 3) Move to the project directory
$ cd advanced-programming-of-cryptographic-methods
\end{verbatim}

\subsection{Configuration}
If you have downloaded the project as a zip archive, you will find a \texttt{.env} already populated
with the necessary configuration values, and an \texttt{EMAIL\_CREDENTIALS.md} file with the email credentials for 
the Gmail account associated to the Resend account specified in the \texttt{.env} file.
On the other hand, if you have set up the environment using the git approach, you need to create a new
\texttt{.env} file in the project root directory with the following configuration:
\begin{verbatim}
# MongoDB Configuration
MONGO_USERNAME=camanager
MONGO_PASSWORD=<choose a safe password>

# AWS/HSM Configuration
AWS_REGION=eu-west-1
AWS_ACCESS_KEY_ID=111122223333
AWS_SECRET_ACCESS_KEY=aaaabbbb11111

# Email Service Configuration (Resend.com)
RESEND_API_KEY=<copy your API key from Resend.com>
RESEND_FROM=onboarding@resend.dev
\end{verbatim}
As the CA is using an emulated version of AWS KMS, the 
\texttt{AWS\_ACCESS\_KEY\_ID} and \texttt{AWS\_SECRET\_ACCESS\_KEY} can be set to the proposed 
test values which emulates real ones. In addition, as the free plan of Resend.com allows sending 
emails only to a single email address, and there are no free domain providers, RESEND\_FROM must be 
set to the default email address provided by Resend.com.

\subsection{Starting the CA}

\textbf{Initial startup:}
\begin{verbatim}
# Build and start all services
docker compose up --build
\end{verbatim}
During the first startup, the system automatically:
\begin{enumerate}
    \item Creates a new root ECDSA key pair in the HSM using curve P256
    \item Generates the root certificate for the CA
    \item Initializes the database schema
    \item Sets up necessary indexes
\end{enumerate}
After that, and in any execution, the system will start the following services:
\begin{itemize}
    \item \textbf{MongoDB}: Database on port 27017
    \item \textbf{Local KMS}: HSM on port 8080
    \item \textbf{Backend}: CA server on port 5000
    \item \textbf{Frontend}: Web interface on port 3000
\end{itemize}

\subsection{Service verification}

\textbf{Health check endpoints:}
\begin{verbatim}
# Backend health check
curl http://localhost:5000/v1/health

# Frontend accessibility
curl http://localhost:3000
\end{verbatim}

\section{Generating a Key Pair with OpenSSL}

To use the system, users must generate a cryptographic key pair. 
The private key should be kept secure, while the public key can be shared with others. 
OpenSSL provides a straightforward way to generate keys from the command line. 
Users can choose among the following algorithms:

\begin{itemize}
    \item \textbf{RSA 2048-bit}
    \begin{verbatim}
openssl genpkey -algorithm RSA -out private_key.pem -pkeyopt rsa_keygen_bits:2048
    \end{verbatim}

    \item \textbf{RSA 4096-bit}
    \begin{verbatim}
openssl genpkey -algorithm RSA -out private_key.pem -pkeyopt rsa_keygen_bits:4096
    \end{verbatim}

    \item \textbf{ECDSA with P-256 curve}
    \begin{verbatim}
openssl ecparam -name prime256v1 -genkey -noout -out private_key.pem
    \end{verbatim}
\end{itemize}
After generating the private key, the corresponding public key can be extracted with:

\begin{verbatim}
openssl pkey -in private_key.pem -pubout -out public_key.pem
\end{verbatim}
Make sure to store \texttt{private\_key.pem} securely, as it grants access to all operations requiring your identity.
