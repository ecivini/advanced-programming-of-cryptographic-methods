\chapter{Requirements}

This chapter defines the requirements for our Certificate Authority, 
categorized into functional and security requirements. Each requirement focuses 
on what the system must accomplish rather than implementation details.

\section{Functional requirements}

The functional requirements define the core capabilities that the Certificate 
Authority must provide to fulfill its role.

\subsection{FR1: Certificate issuance}

The CA must generate and issue digital certificates after verifying the 
identity of the certificate requester. This includes:

\begin{itemize}
    \item Receiving and validating certificate requests
    \item Verifying the authenticity of certificate requests
    \item Generating X.509 compliant certificates
\end{itemize}

Certificate issuance is the foundation of a certificate authority, 
requiring identity verification to prevent unauthorized issuance.

\subsection{FR2: Certificate revocation}

Users must be able to request revocation of their certificates at any time, 
particularly when suspecting private key compromise. The system shall:

\begin{itemize}
    \item Accept authenticated revocation requests from certificate holders
    \item Verify proof of private key ownership before processing revocation
    \item Immediately update the certificate status upon successful verification
\end{itemize}

This capability ensures certificate holders maintain control over their 
digital identities and can respond quickly to security incidents.

\subsection{FR3: Certificate Revocation List (CRL) accessibility}

The CA must maintain and publish a Certificate Revocation List that allows 
anyone to verify certificate status. Requirements include:

\begin{itemize}
    \item Maintaining an up-to-date list of all revoked certificates
    \item Providing public access to CRL information
    \item Digitally signing the CRL to ensure authenticity
\end{itemize}

Public CRL access enables relying parties to perform proper certificate validation.

\subsection{FR4: Identity verification}

The CA must verify both email authenticity and private key ownership for 
certificate applicants. This simplified verification process includes:

\begin{itemize}
    \item Email verification through challenge-response mechanisms
    \item Private key ownership verification via cryptographic challenges
\end{itemize}

This approach provides reasonable assurance that certificate requests come 
from legitimate key-holders controlling the specified email addresses.

\subsection{FR5: Public key publishing}

The CA must publish issued certificates and its own public key to enable 
certificate validation. This requires:

\begin{itemize}
    \item Maintaining a publicly accessible certificate repository
    \item Publishing the CA's public key as a trust anchor
    \item Providing standard interfaces for certificate retrieval
\end{itemize}

Public availability of certificates and the CA public key enables the complete 
certificate validation process for all relying parties.

\subsection{FR6: Certificate renewal}

The CA must support certificate renewal before expiration to ensure service 
continuity. The renewal process shall:

\begin{itemize}
    \item Verify the current certificate has not been revoked
    \item Perform identity verification equivalent to initial issuance
    \item Ensure proper timing to prevent service interruptions
\end{itemize}

Renewal maintains trust relationships beyond individual certificate lifespans 
while preserving security through re-verification.

\subsection{FR7: Cryptographic algorithm support}

The CA must support both RSA and ECDSA cryptographic algorithms to ensure 
broad compatibility:

\begin{itemize}
    \item RSA support with minimum 2048-bit key sizes
    \item ECDSA support using the P-256 curve
    \item Key generation, signing, and verification for both algorithms
\end{itemize}

Multi-algorithm support accommodates diverse client requirements and provides 
flexibility for different security and performance needs.

\section{Security requirements}

Security requirements define the protective measures the Certificate Authority 
must maintain to ensure service integrity, confidentiality, and 
availability.

\subsection{SR1: Certificate authenticity}

Certificates containing valid CA signatures must be considered authentic and 
trustworthy. This requires:

\begin{itemize}
    \item Cryptographically strong signature algorithms
    \item Secure certificate signing procedures
    \item Readily available signature verification mechanisms
\end{itemize}

Certificate authenticity forms the foundation of PKI trust, enabling relying 
parties to confidently validate certificate legitimacy.

\subsection{SR2: Certificate validity verification}

Expired certificates and those appearing in the CRL must be considered invalid 
regardless of signature authenticity. Validation must include:

\begin{itemize}
    \item Expiration date verification for temporal validity
    \item CRL consultation for revocation status
    \item Clear rejection of certificates failing either test
\end{itemize}

Comprehensive validity checking prevents acceptance of certificates that may 
pose security risks despite technical authenticity.

\subsection{SR3: Secure key management}

The CA private key must be protected using Hardware Security Modules (HSMs) 
with the following protections:

\begin{itemize}
    \item Tamper-resistant hardware storage
    \item Protection against unauthorized access and extraction
    \item All signing operations performed within the HSM boundary
    \item Comprehensive audit trails of key usage
\end{itemize}

HSM protection ensures the CA private key, the most critical PKI asset, 
remains secure against various attack vectors.

\subsection{SR4: Data integrity and confidentiality}

All certificate data, configuration, and audit logs must be protected against 
unauthorized access and modification:

\begin{itemize}
    \item Database encryption for stored information
    \item Secure communication protocols for data transmission
    \item Access controls and integrity checking mechanisms
\end{itemize}

Data protection ensures certificate information remains accurate and prevents 
inappropriate disclosure of sensitive information.

These requirements collectively define the functional capabilities and security 
posture necessary for effective and secure Certificate Authority operations.