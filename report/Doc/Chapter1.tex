\chapter{Introduction}
\label{ch:introduction}
This project implements a Certificate Authority (CA) system that manages digital certificates for secure communications. 
A CA is a trusted entity that issues digital certificates to verify the identity of users, servers, or devices in a network.

The system provides a complete solution for certificate management, including issuing new certificates, checking their validity, renewing them before expiration, and revoking them when necessary. 

The implementation focuses on security best practices by using Hardware Security Modules (HSMs) to protect the CA's private key and implementing proper verification procedures.

The system provides user-friendly interfaces through both API endpoints for developers and a web interface for end users. 
Additionally, it ensures system reliability by handling multiple requests efficiently while maintaining data consistency.

\section{Key Features}

The Certificate Authority system supports the complete certificate lifecycle. 
For certificate issuance, the system processes certificate requests, verifies the requester's identity, and issues signed digital certificates. 
It provides certificate validation by checking if certificates are valid, not expired, and haven't been revoked.

The system allows users to renew their certificates before they expire and handles certificate revocation for compromised or no longer needed certificates, maintaining a Certificate Revocation List (CRL). 
Identity verification ensures that certificate requesters own the private key and control the email address associated with their request.

\section{System Architecture}

The system consists of several integrated components. The backend server is a Go-based API that handles all certificate operations and communicates with the database and HSM. 
A Next.js frontend application provides an easy-to-use web interface for certificate management.

Data persistence is handled by MongoDB storage for certificate records, user commitments, and system data. 
Security operations are performed by a Hardware Security Module (HSM) that stores the CA's private key and performs cryptographic operations. 
An automated email service supports identity verification during certificate requests.

\section{Report Structure}

This report provides a comprehensive overview of the Certificate Authority system implementation. 
Chapter 2 presents the requirements analysis and system specifications. 
Chapter 3 details the system design and architecture. 
Chapter 4 covers implementation details and technology choices, while Chapter 5 discusses security considerations and best practices. 
Finally, Chapter 6 provides deployment guidance and usage instructions.

Each chapter builds upon the previous one, providing a complete understanding of the Certificate Authority system from concept to deployment.\cite{coulouris} 
% citazione sennò latex si spacca tutto 