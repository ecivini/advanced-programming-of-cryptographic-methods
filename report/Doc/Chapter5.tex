\chapter{Known limitations}

Despite the project aimed at creating a certificate authority capable of issuing X.509 certificates 
able to verify the identity of the certificate holder, we had to make some compromises due to time
constraints, high cost of services and practicality of the implementation.

\section{Limited identity verification}
The current implementation uses a simplified identity verification process
that only checks email ownership and private key possession, thus lacking a proper identity validation
processes.

\section{Limited number of email addresses}
Due to the use of Resend.com for sending emails, the system is limited to a single email address
for sending challenge emails. This is because the free plan of Resend.com only allows sending emails
from their email address \textit{onboarding@resend.com} and to the email address associated to the Resend.com account.
It is possible to use custom domains to remove this limitation, but we did not find any free solution.

\section{Simplified certificates usage}

The system implements basic X.509 certificate profiles without the full range 
of extensions and policies, for example:
\begin{itemize}
    \item Authority Information Access (AIA) extensions
    \item Subject Alternative Names (SAN) for multiple identities
\end{itemize}

\section{Lack of revocation cause}

While the CA implements a revocation mechanism to revoke certificates, it does not include
reason codes that specify the cause of the revocation events. As a consequence, the system itself
is not able to revoke certificate on behalf of the certificate holder.

\section{System scalability}

The system is implemented as a single-instance application without considering scaling solutions, which
would improve availability and overall security. In particular, a significant improvement would be
to use a clustered HSM deployment, which would allow the CA to continue operating even if one of the HSMs
fails or is under maintenance, and to use threshold cryptography, which would allow the CA to distribute
the signing operations across multiple HSMs.

\section{Standards compliance}

While the system implements main operations for managing X.509 standards, it lacks full compliance 
with advanced PKI standards, in particular tries replicate RFC 5280 and RFC 6960 in a simplified way.

\section{Requiring internet access for all operations}
As our CA is implemented to be real-time, it does provide certificate revocation list files. This means 
that it is not possible to verify the status of a certificate without access to the internet, in contrast 
to what happens within RFC 5280, which allows to download the CRL file beforehand and verify the status 
of a certificate without requiring internet access, since it is able to cache CRL files and use them for 
offline verification.

\section{Single-email address limitation}
As domain providers always require a paid plan to create custom email addresses, the system is limited
to using a single email address for sending emails, which is the default one provided by Resend.com, and
is able to send challenge emails only to the address that is associated to the Resend.com account.