\chapter{Security Considerations}

This chapter critically examines the security architecture of the implemented Certificate 
Authority system, focusing on identifying fundamental vulnerabilities and proposing 
concrete pathways for improvement.


\section{Missing certificate extensions}

The certificate generation process lacks Authority Key Identifier and Subject Key Identifier extensions,
which may cause problems for certificate chain validation. The absence of Certificate Policies extensions 
prevents policy-based validation required in production environments.

This issue can be addressed by fully supporting RFC 5280 \cite{rfc5280} compliance, 
including configurable policy engines, name constraints support, and complete certificate path 
validation logic.


\section{Side channel attacks and information leakage {\color{red}TODO: FIX ISSUE OR DESCRIPTION}}

While Go's crypto library provides constant-time operations for most of its primitives, the CA
relies on database queries and nonce lookup mechanisms that can leak information through timing
patterns. 

Database queries reveal certificate existence through timing patterns, while nonce lookup mechanisms 
use standard hash operations with predictable performance characteristics. Error message patterns
provide another disclosure channel, allowing attackers to distinguish between different failure 
conditions and map internal system architecture.

Resolution requires implementing constant-time operations, normalizing response timing through 
artificial delays, and sanitizing error responses to provide uniform feedback regardless of underlying 
failure conditions.


\section{HSM as a single point of failure}

The current architecture relies on a single HSM instance for all cryptographic operations. 
This design creates a single point of failure for the entire CA, making it inoperable if the HSM 
becomes unavailable due to hardware failure, network connectivity issues, or maintenance requirements.

Furthermore, the HSM represents a single point of compromise where an attacker who gains access to 
the HSM could potentially sign malicious certificates, extract private key material, or manipulate 
the cryptographic operations. 

This issue can be addressed by requiring HSM redundancy through clustered HSM deployments, threshold 
cryptography that distributes signing operations across multiple HSMs, automated failover mechanisms 
that maintain service continuity during HSM maintenance or failure, and comprehensive HSM monitoring 
with real-time availability checking. 

\section{Simplified revocation mechanism}

The revocation mechanism lacks reason codes that specify the cause of the revocation events, 
limiting incident response capabilities. Because of the same reason, the system provides no mechanism 
for CA-initiated revocation, creating problems when certificates need revocation due to external 
threats but certificate holders are unavailable.

This problem can be addressed by extending the revocation mechanism to include reason codes that specify 
the cause of the revocation events, and by implementing a mechanism for the CA to initiate revocation 
when necessary. This would allow the CA to revoke certificates in response to external threats, 
even when certificate holders are unavailable. To prevent censorship, the reason codes should be included
in the OCSP responses, allowing the relying parties to understand the reason for the revocation
without trusting blindly the CA.


\section{Lack of post-quantum cryptography support}

The current implementation lacks post-quantum cryptography support, creating a significant 
long-term security vulnerability as quantum computing technology advances. The system relies entirely 
on ECDSA-SHA256, which will become vulnerable as quantum computers becomes more powerful.

While we initially thought about adding support for post-quantum algorithms, we were limited by the ones 
supported by \textit{local-hsm}, an emulated version of AWS KMS that provides a 
simplified HSM capabilities for development purposes. Unlike production AWS KMS, \textit{local-hsm} does 
not support post-quantum algorithms such as CRYSTALS-Dilithium or CRYSTALS-Kyber, preventing the 
implementation of quantum-resistant cryptography within our development environment. because of this,
the CA cannot generate hybrid certificates that provide both current ECDSA compatibility 
and quantum resistance.

To address this issue, it is enough to use a production HSM that supports post-quantum algorithms, such as AWS KMS,
but unfortunately there is not free plan to use it.

\section{Use of custom JSON-based OCSP responses}

The implementation uses custom JSON-based response formats due to simplicity and speed of development 
for this MVP. These responses try to emulate the ones in RFC 6960 \cite{rfc6960}, implementing a subset of all
specified cases due to time constraints.
Because of this, a significant future improvement would be to use ASN.1 OCSP responses, making this 
implementation more compatible with the already existing PKI ecosystem. Considering that the core 
functionalities of OCSP are already implemented, this change should not require signficant effort.
